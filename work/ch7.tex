% \chapter{Conclusions}
\chapter{Conluzii}
\label{cap:concluzii}


\subsection{Realizări}
S-a realizat prezentarea detaliată a rețelelor neuronale, acestea fiind cele mai capabile modele folosite în domeniul detectării obiectelor și segmentării semantice a imaginilor. Am văzut cum se comportă blocul de construcție a rețelei: \textit{neuronul}, și cum se construiește rețeaua alcătuită din neuroni care sunt organizate în straturi și au "legături'' ponderate. Metodele de învățare forward propagation și backwards propagation au fost prezentate.\newline
După aceasta am văzut cum funcționează rețelele neuronale specifice domeniului viziunii artificiale (straturile speciale a acestora fiind prezentate).\newline
S-a făcut prezentarea comparativă a diferitelor metode de detectare de obiecte, și am construit o rețea bazându-se pe modelul YOLO (You Only Look Once).


\subsection{Îmbunătățiri posibile}
Câteva îmbunătățiri ar putea fi aduse:
\begin{enumerate}
	\item pentru rețeaua dezvoltată s-ar putea implementa un algoritm mai bun de non-maximum suppression. Acest lucru ar îmbunătăți metoda de calculare a costului la faza de antrenare a rețelei, astfel rețeaua ar putea să-și ajusteze ponderile într-un mod care ar duce la acuratețe mai mare (IoU și mAP ar fi mai mari)
	\item o altă îmbunătățire ar fi să facem rețeaua noastră capabilă să facă detectare de obiecte pe un stream video (momentan fiind capabil de a încărca imagini dintr-un fișier)
\end{enumerate}