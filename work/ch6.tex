% \chapter{Tests and Results}
\chapter{Rezultate teoretice și experimentale}
\label{cap:rezultate}
Comparația corectă a diferitelor algoritmi de segmentare semantică a imaginilor și de detecția obiectelor este o sarcină foarte grea; nu se poate stabili care dintre modele este cel mai bun. În aplicațiile din viața reală, alegerea modelului potrivit se face prin balansarea capabilităților ei: viteza de funcționare și acuratețe. Pe lângă alegerile referitoare la detector, câțiva alți factori trebuie luate în considerare când vorbim de comparație:

% enumerate shit which is crucial for making a good object detector
\begin{enumerate}
	\item rezoluția imaginii de intrare
	\item numărul propunerilor generate pentru a fi clasificate
	\item setul de antrenare
	\item extractorul de trăsături folosit
	\item funcția de cost pentru antrenare de localizare
	\item configurarea antrenării (viteza de antrenare a rețelei neuronale, mărimea loturilor de antrenare, micșorarea ratei de învățare)
	\item etc.
\end{enumerate}

Mai rău, technologia evoluează atât de repede încât orice comparație devine încechită destul de repede.\newline
În această parte a lucrării sumarizăm rezultatele din articole individuale, pentru a arăta imaginea de ansamblu a algoritmilor.

\section{Rezultate de Performanță}
În această secțiune sumarizăm performanțele diferitelor modele raportate în articolele corespunzătoare.\newline
Metrica pentru măsurarea performanței în contextul detecției obiectelor este \textit{mAP (mean Average Precision)}. Este definit ca \textit{the average of the maximum precisions at different recall values}, adică media maximelor precizii la diferite valori de rechemare. Pentru a înțelege acest concept, trebuie mai întâi să recapitulăm conceptele de precizie și rechemare mai întâi.
\paragraph{Precizie} măsoară acuratețea predicțiilor, i.e. procentul predicțiilor corecte.
\begin{equation}
Precizie = \frac{TruePositive}{TruePositive + FalsePositive}.
\end{equation}

\paragraph{Recall (rechemare)} măsoară cât de precise sunt identificările pozitive.
\begin{equation}
Recall = \frac{TruePositive}{TruePositive + FalsNegative}.
\end{equation}

\paragraph{AP} \textit{Average Precision} este media a mai multor IoU pentru toate categoriile de obiecte.

Următorul tabel prezintă preciziile diferitelor rețele de detectare de obiecte:

\begin{center}
    \begin{tabular}{| l | l | l | l |}
    \hline
    Method & mAP \\ \hline
    Monday & 11C \\ \hline
    Tuesday & 9C  \\ \hline
    Wednesday & 10C \\  \hline
    \end{tabular}
\end{center}



\subsection{Faster R-CNN}






Împreună cu partea de prezentare a proiectului, trebuie să reprezinte aproximativ 70\% din lucrare. 

Aici sunt prezentate metodele de validare a soluțiilor/sistemului descris în capitolele anterioare, scenariile de testare a corectitudinii funcționale, a utilizabilității, performanței etc.   

Rezultatele testelor experimentale necesită, în general interpretări (dacă rezultatele obținute corespund așteptărilor, intuițiilor cititorului, de ce apar variații/excepții etc.) și comparații cu rezultatele altor metode similare. 

Sistemele de testare și testele propriu-zise trebuie descrise detaliat astfel încât să poată fi reproduse și de alții care poate vor să-și compare soluțiile lor cu a voastră (eventual, codul testelor poate fi pus în anexe). Dacă se poate alegeți pentru evaluarea sistemului vostru benchmark-uri (pachete de testare) dedicate, astfel încât comparația cu alte sisteme să poată fi făcută mai ușor. În plus, astfel de teste sunt mult mai complete și mai realiste decât cele dezvoltate de voi. Oricum, încercați ca testele efectuate să nu fie triviale, ci să acopere scenarii cât mai reale, mai complexe și mai relevante ale funcționării sistemului vostru. 

% \section{Functional Tests}
\section{Teste de funcționalitate}



% \section{Performance Tests}
\section{Teste de performanță}