% \chapter{Project's Objectives and Specification}
\chapter{Obiective și specificații}
\label{cap:obiective-specificatii}
Cum sugerează și titlul, acest capitol va conține o descriere detaliată a temei de cercetare propriu-zisă, precum și obiectivele și specificațiile lucrării de disertație. Titlul lucrării este \textit{\thesistitle}.\newline

\section{Obiective}
În următoarea secție se prizintă scopul lucrării de disertație împreună cu specificațiile funcționale și non-funcționale.\newline
Obiectivul general al acestei lucrări este studiul și analiza metodelor de ultima oră (state-of-the-art) de detecția obiectelor și segmentarea semantică a imaginilor precum și îmbunătățirea rețelei neurale DarkNet YOLO  \cite{DBLP:journals/corr/abs-1804-02767} (You Only Look Once)  pentru ca aceasta să fie potrivită pentru detectarea obiectelor și segmentarea semantică a imaginilor. Obiectivele propuse:
\begin{itemize}
	\item Documentarea extensivă în domeniu
	\item Crearea unui studiu comparativ a metodelor de ultima oră din domeniul detecției obiectelor și a segmentării semantice a imaginilor (pentru fiecare sistem se ia în vedere architectura, ideile unice aplicate în sistemul respectiv și punc
	\item Înțelegerea și analiza problemelor metodelor folosite precum și a sistemului YOLO
	\item Crearea unui plan pentru îmbunătățirea sistemului YOLO și justificarea planului (YOLO atinge numai 57.9\% mAP pe testul COCO - Common Objects in Context)
	\item Documentarea rezultatelor
\end{itemize}

\section{Specificații}
\subsection{Specificații funcționale}
Specificațiile funcționale a sistemului implementat sunt:
\begin{itemize}
	\item Sistemul trebuie să fie capabil să functioneze pentru timp nedefinit (implementarea trebuie să evite memory leak-uri sau alte erori care ar duce la eșuarea rulării pentru timp îndelungat)
	\item Sistemul trebuie să fie capabil să funcționeze în timp real
	\item Sistemul trebuie să fie capabil să "țină pasul" cu o înregistrare video cu rezoluția de 800x600 @30fps
\end{itemize}
\subsection{Specificații non-funcționale}
Specificațiile non-funcționale a sistemului sunt:
\begin{itemize}
	\item fiabilitate
	\item funcționare eficientă (sistemul trebuie să fie capabil să ruleze cu 6 giga de ram și procesor Inter Core i7-3610QM) și efectivă
	\item stabilitate
	\item pentru detecția obiectelor și segmentarea semantică se folosește o rețea neuronală
	\item scorul obținut pe testul COCO pentru detecție să fie mai mare decât  80\%
	\item mentenabilitate
	\item sistemul trebuie să fie capabil să încarce greutățile dintr-o antrenare anterioară, rezultatul căreia a fost salvată
	\item integrabilitate
	\item în acest proiect n-am avut îm vedere portabilitatea sistemului (asta este un future development)
\end{itemize}

\section{Cazuri de Utilizare}
Funcționarea sistemului se poate arăta prin prezentarea cazurilor de utilizare.\newline
În general sunt trei cazuri mari de utilizare, care trebuie să fie funcționale ca să putem zice că proiectul este complet. Acestea vor fi descrise în detaliu în capitolul 5, acuma mulțumindu-ne cu rezumatul lor:
\begin{enumerate}
	\item antrenarea rețelei: un utilizator poate să pornească o sesiune de antrenare pentru rețeaua neuronală din linia de comandă. La sfărșitul procesului parametrii rețelei sunt salvate într-un fișier binar.
	\item testarea rețelei: un utilizator poate să încarce paramtrii de la o rețea deja antrenată dintr-un fișier binar. După aceasta folosing Microsoft COCO Dataset, el poate să înspecteze acuratețea sistemului
	\item folosirea rețelei pentru detecția obiectelor dintr-o imagine sau dintr-o recordare video
\end{enumerate}