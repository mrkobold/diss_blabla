% \chapter{Theoretical Backgound}
\chapter{Fundamente teoretice}
\label{cap:fund-teoretice}

Aici se descriu pe scurt aspecte teoretice pe care se bazează lucrarea. Conținutul acestui capitol trebuie gândit pentru un citor care nu e specializat pe domeniul temei și nu cunoaște chestiunile de bază despre subiect. Pentru un cititor specializat, capitolul poate să stabilească un limbaj comun, relativ la termenii care pot fi interpretați diferit. 

Acest capitol nu trebuie gândit și scris nici ca un copy-paste din alte surse, nici ca zona de reglaj a numărului de pagini ale lucrării. Deși va conține chestiuni pe care le-ați studiat și voi și pe care v-ați bazat, el trebuie să fie o compilare a surselor folosite, care să aibă sens și relevanță pentru lucrarea voastră. Trebuie să fie o descriere coerentă și logică a unor aspecte care ușurează sau fac posibilă înțelegerea părților următoare ale lucrării. Nu trebuie intrat insă prea mult în detalii, ci spuse doar chestiunile esențiale. 

Dacă preluați text, figuri, tabela etc. din sursele de documentare, acestea din urmă trebuie indicate explicit. 

Reprezintă cca. 10\% din lucrare.