f% \chapter{Bibliographic Survey}
\chapter{Studiu bibliografic}
\label{cap:studiu-bibliografic}
În acest capitol se face o prezentare a stagiului în care este domeniul; se detaliază cele mai bune sisteme de detecția obiectelor și segmentarea imaginilor de ultima oră prin informații căpătate din articole și cărți.\newline
Va fi vorbă despre viziunea artificială, rețelele neurale (mai ales architecturile folosite în procesarea imaginilor). Temele detecției obiectelor și segmentării semantice a imaginilor vor fi detaliate, prezentând caracteristicile specifice sistemelor inteligente capabile de a executa aceste sarcini.Dar mai întâi o imagine de ansamblu...\newline
\section{Imaginea de ansamblu}
De multe ori este o confuzie când vorbim despre conceptele de clasificarea imaginilor, detecția obiectelor și segmentarea semantică a imaginilor. După cum se vede în figura \ref{fig:class_detect_semgent} Toți trei fac parte din tematica înțelegerii unei imagini existând suprapuneri între aceștia, dar bineînțeles vorbim despre trei procese care diferă atât prin rezultatele executării cât și prin complexitatea algoritmilor care pot să le realizeze.\newline
Clasificarea imaginilor este procesul prin care după cum zice și numele, este procesul prin care obiectul fotografiat este înscrisă într-o categorie de obiecte (primind o imagine, sistemul trebuie să decidă clasa căreia aparține obiectul e.g. un câine sau o pisică).\newline
Detecția obiectelor pe de altă parte este procesul prin care identificăm locația a mai multor obiecte într-o imagine (sau numărarea instanțelor de un anumit tip de obiect într-o imagine). Majoritatea algoritmilor de detecția obiectelor (precum și algoritmul folosit în această lucrare) funcționează în următorul fel:
\begin{enumerate}
	\item un model sau un algoritm este folosit pentru generarea unei multitudini de regiuni de interest. Regiunile acoperă toată imaginea iar caracteristica lor comună este șansa ridicata de a fi bounding box pentru un obiect
	\item în pasul doi se extrag trăsături specifice din regiunile generate la pasul anterior. Se face o evaluare a trăsăturilor extrase și sistemul determină dacă regiunea respectivă conține un obiect, și dacă da, atunci ce fel de obiect (ce categorie). Acest pas este o componentă de clasificare a imaginilor
	\item ultimul pas este o post procesare, regiunile cu suprapuneri fiind combinate pentru alcătuirea unui bounding box
\end{enumerate}
Segmentarea semantică a unei imagini este procesul prin care imaginea digitală se partiționează în mai multi segmenți (seturi de pixeli cunoscute ca superpixeli). Scopul segmentării semantice este aceea de a simplifica și/sau de a schimba reprezentarea unei imagini pentru o înțelegere și analizare mai simplă. Segmentarea semantică este adesea folosită pentru localizarea obiectelor și a conturilor în imagini. Mai precis, segmentarea semantică a unei imagini este procesul prin care pentru fiecare pixel în asignăm o etichetă, astfel încât pixelii care împărtășesc aceeași etichetă, împărtășesc caracteristici definite (e.g. aparțin aceluiași obiect).
%image presenting image classification, detection and segmentation
\begin{figure}[h!]
    	\centering
	\captionsetup{justification=centering, margin=2cm}
	\includegraphics[width=0.9\textwidth]{figures/class_detect_segment.jpeg}
	\caption{Clasificare de imagine, localizare, detecție de obiecte, segmentare semantică \cite{class_detect_segment}}
	\label{fig:class_detect_semgent}
\end{figure}
\section{Domeniile din care face parte această lucare}
\subsection{Viziunea Artificială}
Viziunea artificială este un domeniu interdisciplinar care se ocupă cu metode cu ajutorul cărora calculatoarele pot să câstige o întelegere la un nivel ridicat și abstract ale imaginilor digitale. Scopul este automatizarea proceselor de care sistemul vizual human este capabil.\newline
Problemele în acest domeniu se pot încadra în mai multe categorii cum ar fi obținerea, procesarea, analiza și întelegerea imaginilor digitale. \textit{Înțelegerea} în acest context înseamnă transformarea imaginilor vizuale în descrieri, care la rândul lor pot fi folosite în alte procese (e.g. luarea deciziilor). După \cite{Szeliski10computervision} viziunea artificială este folosită în următoarele domenii:
\begin{itemize}
	\item controlul proceselor, inspecție automatică
	\item navigare
	\item probleme de clasificare și organizare
	\item detectare de evenimente
	\item recunoaștere de obiecte, identificare (e.g. persounei după amprentă sau față, etc.), detectare (e.g. verificarea prezenței unui obiect pe o imagine)
	\item analiza mișcărilor, reconstruirea scenelor, restorarea imaginilor, etc.
\end{itemize}
\subsubsection{Metode uzuale în domeniul viziunii artificiale}
În \cite{M82computervision} Kusuma Kumari B. M. face o prezentare generală a metodelor des folosite în domeniul viziunii artificiale și procesarea imaginilor.\newline
După ce am obținut o imagine digitală, urmează pasul de \textbf{pre-procesare}. În acest pas putem aplica operații de re-sampling, intensificarea sau micșorarea contrastului, operații pentru a reduce zgomotul din imagine, transformări de culoare (binarizare, grayscale, etc.).\newline
Odată ce imaginea atinge o calitate anume (și putem să avem încredere că algoritmii următori pot să prelucreze imaginea) urmează de obicei pasul de \textbf{extragerea trăsăturilor}. Imaginea este scanată pentru trăsături specifice (astea aflându-se la o granularitate mare sau mică) care împreună definesc un obiect complex. De obicei cautăm linii, muchii, colțuri, puncte, cercuri, creste, etc.\newline
Am ajuns la pasul în care folosind trăsăturile extrase la pasul anterior facem o \textbf{procesare} la nivel înalt. Deja trebuie să lucrăm numai cu trăsăturile extrase, aceasta fiind o mare îmbunătățire față de situația ipotetică în care ar trebui să lucrăm cu imaginea digitală de intrare. Pe baza trasăturilor extrase putem să executăm recunoașteri de obiecte, sau să aplicăm alți algoritmi pe multitudinea trăsăturilor (de exemplu pentru construirea unei imagini noi).\newline
În anumite cazuri mai este un pas, și anume \textbf{luarea deciziilor}. Aici pe baza tuturor pașilor anteriori programul ia o decizie, cu asta împlinind funcționalitatea completă.
\subsection{Rețele Neuronale}
În această secțiune o să fie prezentată ideea de bază a funcționării rețelelor neuronale cu ajutorul unor articole.\newline
Dorffner ș.a. în \cite{dorffner1996neural} prezintă funcționarea de bază a rețelelor neuronale; rețelele neuronale artificiale sunt inspirate de natură, acestea încercând să reproducă funcționarea rețelelor de neuroni din sistemul nervos al animalelor. O rețea neuronală artificială ca și cea "naturală" este construită din neuroni care alcătuiesc straturi de neuroni. Neuronii din două straturi consecutive sunt legate, legătura având o pondere: un număr care exprimă puterea legăturii dintre cei doi neuroni.\newline
În \cite{rojas1996backpropagation} Rojas ș.a. prezintă conceptul care stă la baza antrenării unei rețele neuronale: retropropagarea erorii. 	Acest algoritm pornește de la un \textit{set de date de antrenare} format din perechi intrare - ieșire dorită. Ponderele rețelei se inițializează cu valori aleatorii, alese de obicei din intervalul (-1, 1).\newline
Algoritmul (în cea mai simplă formă a ei) are două stagii:
\begin{enumerate}
	\item rețeaua neuronală primește o intrare din setul de antrenare. Folosind valorile curente ale ponderilor se face propagarea înainte a informației de intrare, calculându-se ieșirea reală furnizată de rețea
	\item ieșirea reală se compară cu valoarea dorită corespunzătoare setului de antrenare si eroarea calculată astfel se propagă înapoi în rețea - de la stratul de ieșire, spre stratul de intrare - pentru modificarea ponderilor.
\end{enumerate}
În \cite{aima} autorii specifică caracteristicile definitive a unei rețele neuronale artificiale, astea fiind:
\begin{itemize}
	\item architectura rețelei este definită de mai multe aspecte: numărul straturilor de neuroni, numărul neuronilor pe fiecare strat, ponderile rețelei, etc.
	\item regula de învățare (learning rule) este mechanismul responsabil pentru ajustarea ponderilor în iterațiile învățării. Alegerea aceste reguli poate diferenția între eșec și succes
	\item regula de activare a unui neuron: ce se întâmplă internal în fiecare neuron? Cum sunt intrările lui prelucrate pentru a obține ieșirea? Regula de activare este descrisă de funcția de activare.
\end{itemize}
\subsection{Rețele neuronale convoluționale}
Pal, Nikhil R and Pal și Sankar K în \cite{pal1993review} prezintă că dintre technicile de segmentarea imaginilor, rețelele neuronale convoluționale întrec celelalte abordări cu o margine semnificativă.\newline
Rețelele neuronale convoluționale sunt foarte similare cu rețelele neuronale profunde clasice: sunt alcătuite din neuroni care au ponderi pe care pot să le învețe. Fiecare neuron are intrări, aplică funcția de activare și propagă mai departe activarea produsă.\newline
Deci ce se schimbă? Architecturile convoluționale se bazează pe asumpția explicită, că intrările sunt \textbf{imagini}. Această asumpție ne împuternicesc să facem o architectură mai avantajoasă. Astfel pasul de propagare înainte devine mai eficientă iar costurile de antrenare se micșorează dat fiind faptul că numărul parametrilor se micșorează grozav.
În \cite{csfromuniversity} se prezintă de ce avem nevoie de îmbunătățirea oferită de rețelele convoluționale: rețelele neuronale regulare nu se scalează bine pentru imagini mari. În datase-ul CIFAR-10 imaginile au numai mărimea de 32x32x3 (32 lățime, 32 înălțime, 3 canale de culori), deci un strat complet conectat ar conține 32*32*3 = 3072 ponderi. Chiar dacă acest număr pare maniabil, se vede clar că structura fully-connected nu se scalează pentru imagini mari. De exemplu o poză color, cu o rezoluție de 200x200 pixeli ar avea nevoie de 200*200*3 = 120,000 ponderi.\newline
Rețelele convoluționale elimină problema numerelor mari prin punerea în practică a două idei: folosirea straturilor convoluționale și a straturilor de pooling. Dintre astea două straturile convoluționale fac marea parte a computărilor. Un strat convoluțional este alcătuit de fapt de în set de filtre; fiecare extrage trăsăturile corespunzătoare lui, și construieste o hartă de activări.\newline
Acum să vorbim despre detecția de obiecte și segmentarea semantică a unei imagini folosind rețele neuronale convoluționale...
\subsection{Detecția Obiectelor și Segmentarea Imaginilor Folosind Rețele Neuronale}
În \cite{historyCNN} Dhruv Parthasarathy prezintă că pentru detecția obiectelor și segmentarea semantică, cele mai bune rezultate sunt produse de rețele neuronale tip R-CNN (Regional Convolutional Neural Network). R-CNN are ca scop găsirea obiectelor principale în imagine (prin identificarea unui bounding box pentru fiecare) precum și clasificarea acestora într-o categorie. Intrarea în R-CNN este o imagine, iar ieșirea este un set de bounding boxes plus lista etichetelor obiectelor din imaginea de la intrare. Dar cum poate să specifice bounding box pentru obiecte? Ideea de bază a R-CNN este foarte intuitivă - se generează o miltitudine de regiuni în imagine și pentru fiecare se verifică cu ce probabilitate se poate clasifica obiectul din interior.\newline
%image showing how selective search proposes regions
\begin{figure}[h!]
    	\centering
	\captionsetup{justification=centering, margin=2cm}
	\includegraphics[width=0.9\textwidth]{figures/selective_search.png}
	\caption{Selective Search \cite{selective_search}}
	\label{fig:selective_search}
\end{figure}
Regiunile astea nu se generează la întâmplare, ci cu ajutorul unui algoritm: Selective Search. Selective search se uită pe imaginea completă, și generează regiunile astfel încât astea să conțină pixelii adiacenți care pot fi grupați împreună, cum ar fi: pixeli cu culoare asemănătoare, texturi asemănătoare, intensitate asemănătoare, etc.\newline

Odată ce regiunile (cu potențial mare de a conține un obiect) sunt generate de către selective search, R-CNN le clasifică cu o versiune modificată a rețelei AlexNet. Pe ultimul strat din rețeaua neuronală, se decide dacă regiunea încercată este într-adevăr un obiect, și dacă răspunsul este da, atunci ce fel de obiect?\newline
Cu toate că R-CNN funcționează coret, are un singur dezavantaj destul de serios, și anume: este foarte lent atât la folosirea propriu zisă cât și la faza de antrenare. Pentru toate ~2000 de regiuni generate cu selective search, se face o propagare înainte prin rețeaua neuronală (AlexNet), ceea ce înseamnă 2000 clasificări per imagine. Acest lucru nu se scalează bine și nu are șansa de a funcționa în timp real pentru captări video.\newline
În anul 2015, Ross Girshick (creatorul R-CNN) a rezolvat ambele probleme prin crearea unui nou algoritm numit Fast R-CNN. Să vedem care sunt ămbunătățirile aduse de acest sistem?
%image how RoI pooling works
\begin{figure}[h!]
    	\centering
	\captionsetup{justification=centering, margin=2cm}
	\includegraphics[width=0.7\textwidth]{figures/fast_rcnn.png}
	\caption{RoI Pooling \cite{historyCNN}}
	\label{fig:roi_pooling}
\end{figure}
\begin{enumerate}
	\item RoI Pooling (Region of Interest Pooling): s-a recunoscut faptul că când se efectuează pasul de propagare înainte pentru cele ~2000 regiuni propuse, sunt foarte multe suprapuneri. Îmbunătățirea constă în realizarea faptei, că în loc de a rula pe toate regiunile o clasificare, am putea să rulăm rețeaua neuronală o dată pe toată imaginea dacă găsim o metodă pentru împărtășirea acestei computatații pentru toate regiunile propuse (\ref{fig:roi_pooling}).
	\item Combinarea tuturor modelelor într-o rețea: în Fast R-CNN rețeaua neuronală convoluțională, clasificatorul și generatorul de regiuni sunt antrenate împreună. Unde în R-CNN am avut modele separate pentru extragerea trăsăturilor cu CNN, pentru clasificare (Support Vector Machine) și pentru generarea regiunilor de interest (Selective Search), acolo în Fast R-CNN se folosește o singură rețea pentru calcularea tuturor. Fast R-CNN a înlocuit clasificatorul SVM cu un strat softmax. În plus a mai adăugat un strat de regresie lineară în paralele cu stratul softmax introdus pentru a crea coordonatele care limitează regiunile.\newline
Chiar și cu toate avansările acestea mai era un bottleneck în procesul Fast R-CNN, și anume generatorul de regiuni. Cum am văzut, primul pas spre detecția locației obiectelor era generarea unei multitudini de locații potențiale (regiunile de interest). În Fast R-CNN acestea sunt generate cu folosirea Selective Search, un proces destul de lent. În anul 2015 o echipă de la Microsoft a găsit o metodă pentru a accelera pocesul de generare a regiunilor. Această metodă (Faster R-CNN) vine aproape fără costuri. Ideea principală a Faster R-CNNului era că propunerile de regiuni de fapt depind de trăsăturile din imagine. Și dacă asta este cazul, de ce să nu refolosim rezultatele CNN-ului și pentru generarea regiunilor?
\end{enumerate}

\subsubsection{Segmentare semantică - Mask R-CNN: Faster R-CNN la nivel de pixel}
În 2017 Kaiming He și o echipă de cercetători de la Facebook au găsit o metodă pentru a extinde Faster R-CNN să fie capabil de segmentare semantică la nivel de pixel. Acest lucru este făcut prin adăugarea unei ramuri capabile de a decide dacă un pixel aparține unui obiect sau nu (aceasta fiind o rețea neuronală complet convoluțională).
%image meaning of semantic segmentation
\begin{figure}[h!]
    	\centering
	\captionsetup{justification=centering, margin=2cm}
	\includegraphics[width=0.5\textwidth]{figures/semantic_segmentation.png}
	\caption{Segmentare semantică \cite{DBLP:journals/corr/HeGDG17}}
	\label{fig:segmentare_semantica}
\end{figure}
Intrarea ramurei de segmentare este o hartă de trăsături, iar ieșirea este o matrice cu 1 în toate pozițiile în care se află pixeli care aparțin obiectului și 0 în rest (o mască binară). Pentru o segmentare cu precizie înaltă, s-a mai făcut o modificare pe Mask R-CNN. Pentru a atinge acest scop, autorii au ajustat RoIPool astfel încât să genereze regiuni mai precise. Algoritmul lor se numește RoIAlign.\newline
Rădăcina problemei era faptul că când se face extragere de trăsături, inevitabil pierdem informație spațială. Să ne imaginăm că avem o imagine de 128x128 pixeli, iar harta de trăsături construită de CNN a acestei imagini este de 25x25 pixeli. Să ne imaginăm, ca vrem trăsăturile din regiunea colțului din stânga sus, cu o dimensiune de 15x15 pe imaginea originală. Cum putem să selectăm pixelii corespunzători din harta de trăsăuri? Știm că fiecare pixel din imaginea originală corespune a ~25/128 pixeli din imaginea construită. Pentru a selecta 15 pixeli din imaginea originală, selectăm 15x25/128~2.93 pixeli.\newline
În RoIPool am rotunji jos acest număr și am alege 2 pixeli, cauzând o nealiniere mică. În schimb în RoIAlign se folosește interpolare biliniară penru a găsi răspunsul la întrebarea: \textit{ce se află la pixelul 2.93?}
\begin{figure}[h!]
    	\centering
	\captionsetup{justification=centering, margin=2cm}
	\includegraphics[width=0.9\textwidth]{figures/image_segmentation_results_mrcnn.png}
	\caption{Segmentare semantică cu Mask R-CNN \cite{DBLP:journals/corr/HeGDG17}}
	\label{fig:segmentare_semantica cu Mask R-CNN}
\end{figure}