% \chapter{Bibliographic Survey}
\chapter{Studiu bibliografic}
\label{cap:studiu-bibliografic}

Documentarea bibliografică are ca obiectiv fixarea referențialului în care se situează tema, prezentarea susrselor bibliografice utilizate și a cercetărilor similare și raportarea abordării din lucrare la acestea.

Referințele bibliografice se vor face pentru fiecare carte, articol sau material folosit pentru elaborarea lucrării de licență. 

Reprezintă cca. 10--15\% din lucrare.


% \section{Related Work}
\section{Abordări similare}

Comparați abordarea voastră cu cele ale altor soluții: ce e asemănător, ce e diferit (și, de preferat, mai bun). 

Citarea referințelor se face ca în exemplele \ref{subsec:s10} din Bibliografie. 
Vezi citările următoare.

În articolul \cite{Antoniou04} autorul descrie configurația tehnică a unei "honeynet" și prezintă câteva atacuri de actualitate asupra honeynet, precum și o serie de recomandări pentru securizarea sistemelor conectate la rețele de calculatoare.

% În capitolul 4 al [], referitor la valoare honeypots, Spitzner prezintă avantajele și dezavantajele acestora.

În articolul on-line \cite{electronic-citation} găsim detalii interesante despre \dots.


% \section{Technologies and Methods}
\section{Tehnici/Tehnologii/Surse folosite}

Sursele de documentare referitoare la metodele, tehnologiile, ideile folosite. 



